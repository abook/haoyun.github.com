\documentclass{simplenotes}
\topic{Partial Differential Equations}
\season{Fall 2012}
\date{September 4, 2012}
\speaker{Yun Hao}
\title{Lecture 1}



%=========
\DeclareMathOperator{\Curl}{curl}
\DeclareMathOperator{\Div}{div}
\DeclareMathOperator{\Tr}{tr}
\renewcommand{\phi}{\varphi}
\renewcommand{\rho}{\varrho}
\newtheorem{exercise}{Exercise}

\begin{document}

\maketitle

\section{Introduction}

\begin{definition}[Partial Differential Equations (PDEs)]
...
\end{definition}

However we only concentrate on those questions that natually occur in varaous applications, such as physics and other sciences, engineering and economics.

Now let's give some example of PDEs.

\begin{example}
\begin{align}
&\Delta u:= \sum_{i=1}^d u_{x^ix^i} =0 \tag{Laplace Equation}\\
&\Delta u =f \tag{Poisson Equation}\\
&u_t =\Delta u \tag{Heat Equation}\\
&u_{tt} =\Delta u \tag{Wave Equation}\\
&(1+u_y^2)u_{xx}-2u_xu_yu_{xy}+(1+u_x^2)u_{yy}=0 \tag{Minimal Surface Equation}\\
&\begin{cases}
\Div B =0\\
B_t - \Curl B =0\\
\Div E = 4\pi\rho\\
E_t-\Curl E= -4\pi j
\end{cases}\tag{Maxwell Equations}\\
&\det(u_{x^ix^j})=f \tag{Monge-Ampere Equation}\\
&u_t-6uu_x+u_{xxx}=0 \tag{Koreweg-de Vires Equation}
\end{align}
\end{example}

\subsection{Clssification}
\paragraph{By of linear of nonlinear}

\begin{itemize}[nosep]
\item Linear equations
\item Non-linear equations
\begin{itemize}[nosep]
\item
Quasilinear equations
\item
Sminilinear equations
\end{itemize}
\end{itemize}

\paragraph{By order}
\begin{itemize}[nosep]
\item First order
\item Second order
\begin{itemize}[nosep]
\item
elliptic
\item
parabolic
\item
hyperbolic
\item
etc.
\end{itemize}
\item Higher order
\end{itemize}

\begin{exercise}
Classify all the equations above.
\end{exercise}

\begin{exercise}
Give more equations and try to classify them according to the standard given above.
\end{exercise}

\section{What will we learn form this textbook}

\paragraph{Question 1}
How to find a solution of elliptic PDES.

In general we have the following five method:
\begin{itemize}
\item[0] Write down an \emph{eplicit} formula for the solution in terms of the given data.

具有极大的局限性。仅仅对少数方程有效。
\item[I] Solve a sequence of auxiliary problems that \emph{approximate} the given one and show ...

偏微分方程可以看作一个无穷维函数空间上的函数方程,从而使用有限维空间逼近。

\item[II] Start any wher, with the required constrains satisfied, and let things \emph{flow} toward a solution.

解是一个扩散过程的渐进平衡态。

\item[III] Solve an \emph{optimization} problem, and ...

变分问题的极值问题,既是偏微分方程的一个来源,也是种解方程的方法。

\item[IV] \emph{Connect} what you want to know to what you already know, that is the \emph{continuity method}.

由已知问题连续的过渡到未知问题。
\end{itemize}

\paragraph{Question 2}
Regularity of Solutions and Uniqueness of Solutions.

generalized solutions  vs. soomth solutions

\paragraph{}
The understanding of the above summary will be deepen after our study. 


\begin{quotation}
In any case, it's usually not very efficient to read a mathematical textbook linearly, and the reader should rather try first to grasp the central statements.
\end{quotation}

可以参考 MIT 公开课程的教学计划。地址见:\url{haoyun.github.com/academic/seminars/2012fall-pde/}

\section{Hamonic Fucntions}

\begin{definition}[Harmonic functions]
A function  $u$ on a domain $\Omega$ that satisfies the Laplace Equation:
\[\Delta u = 0\]
is called a harmonic function on $\Omega$.
\end{definition}

\begin{remark}
All harmonic functions constitue a linear space. Simple varification will prove this. A higher view of point is that all harmonic fuctions is the kernel of Laplace operator, which is a linear operator.
\end{remark}

\begin{example}
Real and imaginary part of an complex analytic function.
(As an exercise, give some examples, and note that the derivative of an analytic functions is also analytic, thus we can find even more desired functions.) This is a consequence of Cauthy-Riemann Equation.
\end{example}

\begin{remark}
This view of point enable us to solve some problems on harmonic functions using the mathod and theories developped in Complex Analysis. We will give some example later.
\end{remark}

\begin{lemma}
Laplace operator is rotation-invariant.
\begin{proof}
Note that $\Delta u(x) = \Tr(H[u(x)])$, where $H[u(x)]$ is the Hessein matrix of u(x).
\[\Delta_x u(A(x-y)+y)=\Tr(A^T H[u(A(x-y)+y)] A)\]
\end{proof}
Another proof?
\end{lemma}

Taking into consideration that $\Delta$ is rotation-invariant, we may suppose that the solution of $\Delta u$ is also rotation-invariant, or radical, i.e.,
\[u(x) = \phi(r), \text{ where } r=|x-y| \text{ with fixed } y\]
Then $\Delta u =0$ can be written as 
\[\phi '' (r) +\frac{d-1}{r} \phi(r) =0\]
Solve the second order ordinary differential eqution we obtain that 
\[\phi'(r) =Cr^{1-d}\]
and then
\[\phi(r) =\begin{cases}
C\log r & d=2\\
\frac{C}{2-d}r^{2-d} &d>r
\end{cases}
\]
Where C is a constant.

\begin{definition}[Fundamental Solution]
...
\end{definition}

\begin{lemma}[Divergence Theorem]
...
\end{lemma}


\begin{lemma}[Green's Formula]
...
\end{lemma}

\begin{lemma}[Lebesgue Dominated Convergence Theorem]
...
\end{lemma}

\begin{lemma}[Lebesgue Theorem]
...
\end{lemma}

\begin{theorem}[Green Representation Formula]
\[u(y) = \int_{\partial \Omega} \left( u \frac{\partial \Gamma}{\partial\nu}- \Gamma\frac{\partial u}{\partial\nu} \right )+\int_\Omega \Gamma \Delta u\]
\end{theorem}

\begin{remark}
Form Green Representation Formula we know that $u$ is determined by its normal derivatives on $\partial\Omega$, prvided $\Delta u$ is given, in particular $u$ is harmonic.
\end{remark}


\begin{remark}
Conversely, we cannot, however, obtain a harmonic fuction using this formula by giving an arbitary $\frac{\partial u}{\partial \nu}|_{\partial \Omega}$, because it must satisfy some constrains, say, at least,
\[\int_{\partial \Omega}\frac{\partial u}{\partial \nu} =\int_\Omega \Delta u =0\]
\end{remark}

We now turn to another method.

\begin{definition}[Green Fucntion]
...
\end{definition}

\begin{remark}
The idea of why we introduce Green function si unknown from this textbook.
\end{remark}

We suppose there do exist a Green function for $\Omega$, then by Green Representation Formula, we have

\begin{equation}
u = \int_{\partial \Omega} u \frac{\partial G}{\partial \nu} + \int_\Omega G\Delta u\label{eq:green2}
\end{equation}

\begin{remark}
From (\ref{eq:green2}) we know that $u$ is determined by its value on the boundary, prvided $\Delta u$ is given, in particular $u$ is harmonic.
\end{remark}

\begin{remark}
Under what condition can we  reconstruct $u$ by given $t|_{\partial \Omega}$ and $\Delta u$ is the next question.

...
\end{remark}

Next, we turn to the Dirichlet Problem on a ball.


\begin{equation}
G(x,y):=\begin{cases}
\Gamma(|x-y|)-\Gamma(\frac{|y|}{R}|x-\bar y|) & y\neq 0\\
\Gamma(|x|)-\Gamma(R)& y= 0
\end{cases}
\end{equation}


First of all, $G-\Gamma$ is harmonic

Secondly,  ... $G(R)=0$

Therefore $G(x,y)$ is a Green function of $B(0,R)$.

Next we caculate $\frac{\partial G}{\partial \nu}$, ..., we have 
\begin{equation}
\frac{\partial G}{\partial \nu_x}=...=\frac{R^2-|y|^2}{d\omega_dR}\frac1{|x-y|^d}
\end{equation}
Then we can obtain the following theorem.

\begin{theorem}[Poisson Representation Formula]
...
\begin{proof}

the fact that $\Delta$  commutes with $\int$ leads to $u$ is harmonic about $y$.

Next we show the continuity ...
\end{proof}
\end{theorem}

\begin{corollary}[smoothness]
u is real analytic on $\Omega$.
\begin{proof}
...
\end{proof}
\end{corollary}

\begin{corollary}[uniqueness]
...
\begin{proof}
...
\end{proof}
\end{corollary}


\end{document}
